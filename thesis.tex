% !TEX TS-program = arara
% arara: clean: {files: [ thesis.aux, thesis.bbl, thesis.toc]}
% arara: xelatex
% arara: nomencl
% arara: biber
% arara: biber
% arara: xelatex
% arara: xelatex
% arara: clean: {files: [thesis.aux, thesis.bbl, thesis.toc, thesis.nls, thesis.nlo, thesis.ilg, thesis.bcf, thesis.blg, thesis.out, thesis.run.xml]}
%
%
%
\documentclass[german,12pt,oneside,titlepage,listof=totoc,bibliography=totoc]{scrartcl}
\usepackage[a4paper, left=4cm, right=2cm, top=4cm, bottom=2cm]{geometry}
%====================================
% Layout Pakete
%====================================
\usepackage{float}
\usepackage{fancyhdr}
\usepackage{fancybox}
\usepackage{setspace}
%====================================
% Sprachsupport Pakete
%====================================
\usepackage[ngerman]{babel}
\usepackage[T1]{fontenc}
\usepackage{polyglossia}
\setdefaultlanguage[spelling=new]{german}
\usepackage[german=quotes,autostyle]{csquotes}
%====================================
% Divers
%====================================
\usepackage{threeparttable}
\usepackage{longtable}
\usepackage{pdfpages}
\usepackage{appendix}
\usepackage{pdfsync}
\usepackage{blindtext}
%====================================
% Systemschriftarten Pakete
%====================================
% eine Times new Roman Version
%====================================
\usepackage{fontspec}                   % muss vor den Schriften ausgeführt werden
\setmainfont{Times New Roman}
\defaultfontfeatures[\rmfamily,\sffamily]{Ligatures=TeX}
\addtokomafont{sectionentry}{\mdseries} % keine Fettschriften im Inhaltsverzeichnis
\setkomafont{sectioning}{\bfseries}     % Überschriften mit Serifenschrift
\usepackage{anyfontsize}
\usepackage{ragged2e}
%====================================
% Unter- und Überschriften Pakete
%====================================
\usepackage{caption}[2008/08/24]
\captionsetup[figure]{%
	labelfont={bf,rm},
	textfont={bf,rm},
	justification=raggedright,
	singlelinecheck=off
}
\captionsetup[table]{%
	labelfont={bf,rm},
	textfont={bf,rm},
	justification=raggedright,
	singlelinecheck=off
}
\newcommand{\capquelle}[1]{%
	\par\parbox{\captionwidth}{\raggedright\bigskip Quelle: #1}%
}
\usepackage{hvfloat}
%====================================
% Literaturverzeichnis Art
%====================================
%%%% Neuer Leitfaden (2018)
\usepackage[
	backend=biber,
	style=ext-authoryear,
	maxcitenames=3,	% mindestens 3 Namen ausgeben bevor et. al. kommt
	maxbibnames=999,
	mergedate=false,
	date=iso,
	seconds=true, %werden nicht verwendet, so werden aber Warnungen unterdrückt.
	urldate=iso,
	innamebeforetitle,
	dashed=false,
	autocite=footnote,
	doi=false,
	useprefix=true, % 'von' im Namen beachten (beim Anzeigen)
	mincrossrefs=1
]{biblatex}%iso dateformat für YYYY-MM-DD
\input{app/skripte/modsBiblatex2018}
%====================================
% Fußnoten Pakete
%====================================
\usepackage{footnote}
\usepackage[hang,multiple,bottom,stable]{footmisc} % Mehrere Fußnoten nacheinander mit Komma separiert
\setlength{\footnotesep}{2pt} % Legt den Abstand zwischen zwei Fußnotentexten am unteren Seitenrand fest
\setlength{\footheight}{22cm} % Legt die Höhe des Leerraumes fest am unteren Seitenrand für eine Fußzeile reserviert wird
\setlength{\footnotemargin}{.8em} % Abstand Text Fußnotennummer
\interfootnotelinepenalty=9999 % kein Zeilenumbruch der Fußnote !!!!


\usepackage{fnpct}
\AdaptNoteOpt\footcite\multfootcite
%====================================
% Aufzählungen Pakete
%====================================
\usepackage{enumitem}
\renewcommand{\labelitemi}{$\bullet$}
\renewcommand{\labelitemii}{$\bullet$}
\renewcommand{\labelitemiii}{$\bullet$}
\renewcommand{\labelitemiv}{$\bullet$}
%====================================
% Tabellen Pakete
%====================================
%\usepackage[table]{xcolor}
\usepackage{nicefrac} % Brüche
\usepackage{multirow}
\usepackage{colortbl}
\usepackage{mdwlist}
%\captionsetup[table]{position=above}
%====================================
% Pakete zu Links
%====================================
% keine farbigen Link-Rahmen
%\usepackage[draft=true]{hyperref}
\usepackage[hidelinks]{hyperref}
\urlstyle{rm}
%====================================
% Mathe Formeln etc.
%====================================
\usepackage{newtxmath}
%====================================
% Farben
%====================================
\definecolor{dunkelgrau}{rgb}{0.8,0.8,0.8}
\definecolor{hellgrau}{rgb}{0.0,0.7,0.99}
\definecolor{mauve}{rgb}{0.58,0,0.82}       % Colors for listings
\definecolor{dkgreen}{rgb}{0,0.6,0}         % Colors for listings
%====================================
% sauber formatierter Quelltext
%====================================
\usepackage{listings}
\lstset{
	basicstyle=\small\ttfamily,             % font style
	numberstyle=\small,
	numbers=left,
	numbersep=8pt,                          % distance number text
	stepnumber=1,
	breaklines=true,
	showstringspaces=false,
	frame=l,
	framexleftmargin=1pt,
	framexrightmargin=5pt,
	keywordstyle=\color{blue},              % keyword style
	commentstyle=\color{dkgreen},           % comment style
	stringstyle=\color{mauve}               % string literal style
} % use \lstset{language=Java}
%====================================
% Zeilenabstand 1,5-zeilig
%====================================
\linespread{1.5}
%====================================
% Absatzeinzug und -abstand
% Absätze durch eine neue Zeile
%====================================
\setlength{\parindent}{0mm}
\setlength{\parskip}{0.3em plus 0.5em minus 0.6em}
%====================================
% Skriptdateien Einbinden
%====================================
\addbibresource{deine_inhalte/citations.bib} 
\addbibresource{deine_inhalte/citations_manual.bib} 
%====================================
\graphicspath{{deine_inhalte/Abbildungen/}{deine_inhalte/Titelseite/}{deine_inhalte/Kapitelanhang/}}
\input{app/skripte/weitereEbene}
%====================================
% Abkürzungsverzeichnis
%====================================
\usepackage[intoc]{nomencl}
\renewcommand{\nomname}{Abkürzungsverzeichnis}
\setlength{\nomlabelwidth}{.50\textwidth}
\renewcommand{\nomlabel}[1]{#1 \dotfill}
\setlength{\nomitemsep}{-\parsep}
\makenomenclature
%========================================================================
% Metainformationen
% Hier gibts du deine Daten ein!!!
%========================================================================
% Name der Hochschule
\newcommand{\myHochschulName}{FOM Hochschule für Oekonomie \& Management}

% Standort der Hochschule
\newcommand{\myHochschulStandort}{München}

% Art der Arbeit
%\newcommand{\myThesisArt}{Scientific Essay}
%\newcommand{\myThesisArt}{Bachelor Thesis}
\newcommand{\myThesisArt}{Hausarbeit}

% Studiengang
\newcommand{\myStudiengang}{Wirtschaftsinformatik}

% Zu erlangender akademische Grad
\newcommand{\myAkademischerGrad}{Bachelor of Science (B. Sc.)}

% Titel der Arbeit
\newcommand{\myTitel}{FOM TeX Template Microservice}

% Autor
\newcommand{\myAutorEins}{Fiona Fommie}

%BetreuerArt
% \newcommand{\myAkademischerGrad}{Erstgutachter}
\newcommand{\myBetreuerArt}{Betreuer}

% Betreuer
\newcommand{\myBetreuer}{Prof. X. Xavier}

% Matrikelnummer
\newcommand{\myMatrikelNrEins}{55444111}

% Datum der Abgabe
\newcommand{\myAbgabeDatum}{\today}

%====================================
% Kopfzeile / Header definieren
% Fußzeile / Footer definieren
% Seitenzahlen definieren
%====================================
\fancypagestyle{plain}%
	\fancyhf{} %clear header and footer fields %redefine plain style
	\fancyhead[C]{\thepage}
	\fancyfoot{}
	\renewcommand{\headrulewidth}{0pt}
	\renewcommand{\footrulewidth}{0pt} 
\pagestyle{fancy} %define site style
	\fancyhf{} 
	\fancyhead[C]{\thepage}
	\fancyfoot{}
	%\chead{\thepage}
	\renewcommand{\headrulewidth}{0pt}
	\renewcommand{\footrulewidth}{0pt}

%====================================
% Ab hier startet dein Dokument!!!
%====================================
\begin{document}
	\renewcommand{\figurename}{Abbildung} 			% Bildunterschrift
	\pagenumbering{Roman}							% Seitennumerierung auf römisch umstellen
\renewcommand{\refname}{Literaturverzeichnis}		% "Literaturverzeichnis" benennen
\newcolumntype{C}{>{\arraybackslash}X}				% Neuer Tabellen-Spalten-Typ: Zentriert und umbrechbar
%========================================================================
% Titlelseite Stand: Dezember 2016
%========================================================================
 \begin{titlepage}
    \newgeometry{left=2cm, right=2cm, top=.5cm, bottom=2cm}
    \begin{center}
        \vspace{.5cm}
        \includegraphics[width=3cm]{fom_logo} \\
        \vspace{.5cm}
        \textbf{\LARGE \myHochschulName}\\
        \vspace{.5cm}
        \textmd{\Large \myHochschulStandort}\\
        \vspace{.5cm}
        \textmd{\normalsize Berufsbegleitender Studiengang}\\
        \textmd{\normalsize \myStudiengang}
        %\textmd{\normalsize , \mySemesterZahl. Semester}
        \vspace{1.5cm}\\
        \textbf{\LARGE \myThesisArt}\\
        \vspace{1.5cm}
        \textmd{zur Erlangung des Grades eines}\\
        \textmd{\Large \myAkademischerGrad}\\
        \vspace{1.5cm}
        \textmd{\normalsize im Rahmen der Lehrveranstaltung}\\
        \textmd{\Large \myLehrveranstaltung}\\
        \vspace{1.5cm}
        \textmd{\normalsize über das Thema}\\
        \textbf{\LARGE \myTitel}\\
        \vspace{0.2cm}
    \end{center}
    \normalsize
    \vfill
    \begin{tabular*}{0.50\textwidth}{@{\extracolsep{\fill}}p{4cm}l}
        Betreuender Dozent: & \myBetreuer \\
        \ &
        \\
        Autor: & \myAutorEins
        \\
        Matrikelnummer: & \myMatrikelNrEins
        \\
        Adresse: & \myAdresseEins
        \\
        \ & \myStadtEins
        \\
        Abgabe: & \myAbgabeDatum
        \\
    \end{tabular*}
\end{titlepage}
 \newpage
%====================================
% Definition des Dokuments
% (nach Titelseite)
%====================================
\restoregeometry %Wiederherstellen der Einstellungen der Geometrie aus der Präambel
%========================================================================
% Dokumentseiten ab hier bindest du die folgenden Seiten ein!!!
% Kommentier sie mit - % - um die Seiten auszuschliessen
%========================================================================
% Sperrvermerk Seite
%====================================
% \addcontentsline{toc}{section}{Sperrvermerk}
\section*{Sperrvermerk}
Die vorliegende Arbeit enthält unternehmensinterne Daten des in der Arbeit genannten Unternehmens. Aus diesem Grund ist die Veröffentlichung der Arbeit nicht erlaubt. Die Arbeit darf außerhalb der Hochschule nur mit der ausdrücklichen Genehmigung des Unternehmens veröffentlicht werden.

\vspace{5cm}

\begin{figure}[h]

    \begin{table}[H]
        \centering
        \begin{tabular*}{\textwidth}{c @{\extracolsep{\fill}} ccccc}
            München, \the\day.\the\month.\the\year
            &
            % Bilder mit transparentem Hintergrund können teils zu Problemen führen
            \includegraphics[width=0.35\textwidth]{Unterschrift}\vspace*{-0.35cm}
            \\
            \rule[0.5ex]{12em}{0.55pt} & \rule[0.5ex]{12em}{0.55pt} \\
            (Ort, Datum) & (Eigenhändige Unterschrift)
            \\
        \end{tabular*} \\
    \end{table}

\end{figure}
%====================================
% Inhaltsverzeichnis
%====================================
\setcounter{page}{2} %Zähler manuell hochsetzen
\tableofcontents
\newpage
%====================================
% Abkürzungsverzeichnis
%====================================
\printnomenclature
\newpage
%====================================
% Abbildungsverzeichnis
%====================================
\listoffigures
\newpage
%====================================
% Tabellenverzeichnis
%====================================
%\listoftables
%\newpage
%====================================
% Seitennummerierung auf arabisch und
% ab 1 beginnend umstellen
%====================================
\pagenumbering{arabic}
\setcounter{page}{1}
%====================================
% Deine Kapitel einbinden!!!
% Deine Inhalte einbinden!!!
%====================================

	\section{Einleitung}
\subsection{Einführung in die Thematik}
\blindtext\nomenclature{W3C}{World Wide Web Consortium}
\blindtext\footcite[Vgl. ][]{mswpf}\footcite[Vgl. ][19]{sadtler_rechtskonformes_2017}

\subsection{Problemstellung und Zielsetzung}
\blindtext

\subsection{Methodischer Aufbau der Arbeit}
\blindtext

\section{Hauptteil}
\subsection{Grundlagen der Wirtschaftsinformatik}
„Dies ist ein direktes Zitat."\footcite[][224]{mertens_digitalisierung_2017} \blindtext\footcite[Vgl. ][]{msdatabind}\footcite[Vgl. ][]{lambda}\footcite[Vgl. ][34]{Digitaloekonomie}
\blindenumerate
\blindtext\footcite[Vgl. ][415-426]{Tanenbaum2016}\footcite[Vgl. ][223]{mandl_internet_2019}

\begin{figure}[!htb]
    \caption{Terminal}
    \includegraphics[width=1\textwidth]{.github/terminal}
    \captionsetup{width=1\textwidth}
    \capquelle{\cite[][200]{bsp}}\label{abb_bsp}
\end{figure}
\blindtext

\subsubsection{Design-Prinzip der Separierung von Verantwortlichkeiten}
\blindtext\footcite[Vgl. ][79]{Schelinski2019}

\begin{table}[!htb]
    \centering
    \begin{threeparttable}
        \centering
        \caption{Tabelle}
        \begin{tabular}{cccccccc}
            {$m$} & {$\Re\{\underline{\mathfrak{X}}(m)\}$} & {$-\Im\{\underline{\mathfrak{X}}(m)\}$} & {$\mathfrak{X}(m)$} & {$\frac{\mathfrak{X}(m)}{23}$} & {$A_m$} & {$\varphi(m)\ /\ ^{\circ}$} & {$\varphi_m\ /\ ^{\circ}$} \\
            \hline
            1  & 16.128 & +8.872 & 16.128 & 1.402 & 1.373 & -146.6 & -137.6 \\
            2  & 3.442  & -2.509 & 3.442  & 0.299 & 0.343 & 133.2  & 152.4  \\
            3  & 1.826  & -0.363 & 1.826  & 0.159 & 0.119 & 168.5  & -161.1 \\
            4  & 0.993  & -0.429 & 0.993  & 0.086 & 0.08  & 25.6   & 90     \\
            5  & 1.29   & +0.099 & 1.29   & 0.112 & 0.097 & -175.6 & -114.7 \\
            6  & 0.483  & -0.183 & 0.483  & 0.042 & 0.063 & 22.3   & 122.5  \\
            7  & 0.766  & -0.475 & 0.766  & 0.067 & 0.039 & 141.6  & -122   \\
            8  & 0.624  & +0.365 & 0.624  & 0.054 & 0.04  & -35.7  & 90     \\
            9  & 0.641  & -0.466 & 0.641  & 0.056 & 0.045 & 133.3  & -106.3 \\
            10 & 0.45   & +0.421 & 0.45   & 0.039 & 0.034 & -69.4  & 110.9  \\
            11 & 0.598  & -0.597 & 0.598  & 0.052 & 0.025 & 92.3   & -109.3 \\
            \hline
        \end{tabular}
        \begin{tablenotes}[flushleft]
        \item \normalsize{Quelle: \cite[][200]{bsp}}
        \end{tablenotes}
    \end{threeparttable}
\end{table}

\Blindtext\footcite[Vgl. ][34]{Digitaloekonomie}\footcite[Vgl. ][]{mesh}
\blinditemize
\blindtext\footcite[Vgl. ][511]{Tanenbaum2016}

\subsubsection{LaTeX-Befehle}
    \textnormal{Normale Schrift} 
    \textbullet\addspace \textbf{fette Schrift} 
    \textbullet\addspace \textit{kursive Schrift} 
    \textbullet\addspace \textsl{schiefe Schrift} 
    \textbullet\addspace \underline{unterstrichen} 
    \textbullet\addspace \texttt{Schreib\-ma\-schi\-ne} 
    \textbullet\addspace \textsf{Sans Serif} 
    \textbullet\addspace \textrm{Roman} 

\subsection{Test: Silbentrennung - Gegenüberstellung - darf nicht in den weißen Rand laufen}
Im zweiten Kapitel wird zunächst der Begriff \textit{Bring Your Own Device} (BYOD) definiert. Darauffolgend wird dargelegt welche Fokusse gesetzt werden sollen und Schlussfolgerungen gezogen, sowie eine Sicherheitsrichtlinie abgeleitet. Welche als Grundlage einer Gegenüberstellung mit etablierten Konzepten dient, sodass Strategien erörtert werden können. 

\section{Test der Ebenen}
\subsection{Ebene 2}
\blindtext
\subsection{Ebene 2.2}
\blindtext
\subsubsection{Ebene 3}
\blindtext
\subsubsection{Ebene 3.2}
\blindtext
\subsubsubsection{Ebene 4}
\blindtext
\subsubsubsection{Ebene 4.2}
\blindtext

\section{Schluss}
\subsection{Fazit}
\blindtext (vgl. Abbildung \ref{abb_bsp}).

\subsection{Ausblick}
\blindtext



%====================================
% Literaturverzeichnis
%====================================
\newpage
\printbibliography[nottype=online,heading=bibintoc,title={Literaturverzeichnis}]
\printbibliography[type=online,title={Internetquellen}]
%====================================
% Deine Kapitel einbinden!!!
% Deine Inhalte einbinden!!!
%====================================
\newpage
	\input{deine_inhalte/Kapitelanhang/Erklaerung}


%====================================

\end{document}
